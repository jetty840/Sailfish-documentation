\NextFile{update.html}

\chapter{Updating Sailfish} \label{chap:update}
\index{Updating}

From time to time, new releases of Sailfish are made available.  Announcements
of these releases along with release notes are published by the Sailfish team
to the jetty-firmware interest group at
groups.google.com.\index{Mailing list}\index{New releases}\footnote{Sailfish
was originally named the ``Jetty Firmware'' and hence the name of the
interest group, jetty-firmware.}  These updates usually include new features,
performance enhancements, and bug fixes.  They have generally been field tested
for several months beforehand by the Sailfish community; only after they have
gone through several months of testing by many volunteers around the world
are they ``officially'' released as non-field test releases.  The field test
releases are kept separate from the official releases using a different
download URL.

At such a time that you wish to update Sailfish, check the release notes
to see if there is any special steps you may need to do.  Release notes are
archived online at

\begin{quote}
\myhref{http://jettyfirmware.yolasite.com/}{http://jettyfirmware.yolasite.com/}.\index{Release notes}
\end{quote}

In order to update Sailfish, you will need ReplicatorG 40 -- Sailfish.  Refer
to Section~\ref{sec:soft-reqs} for information
on obtaining this version of ReplicatorG.  Additionally, ReplicatorG must be
configured with the Sailfish download location as per
Section~\ref{sec:download-url}.

\NextFile{update-step-1.html}

\section{Step 1: Obtaining the latest firmware}

When ReplicatorG runs and you are connected to the Internet (and not
behind a firewall proxy), it will check for firmware updates and
automatically download them.  It will log this activity in its logging
region at the bottom of its main window.  You can also see if
ReplicatorG has the firmware you want via the ``Upload new
firmware...'' item of its ``Machine'' menu: select the printer type,
click the ``Next'' button, and see which firmware versions are listed.

\NextFile{update-step-2.html}

\section{Step 2: Before applying an update}

Typically, there is nothing you need to do before updating.  If you are
cautious by nature, you may wish to save your current onboard parameters
to an SD card file.  Do that via the ``EEPROM'' menu of the ``Utilities''
menu.  See Section~\ref{sec:eeprom} for details.
However, this step is not needed.

\NextFile{update-step-3.html}

\section{Step 3: Update}

Once you know that you have the desired firmware downloaded, install it.  The directions for installation are those of
Section~\ref{sec:really-install}.  Follow
those directions.  Once the new version of Sailfish is installed, your
printer will reboot itself.  You should see the new version number displayed
in the ``splash'' screen (Section \ref{sec:Splash}) when the printer reboots.  You can also check the
version information via the ``Version Information'' item of the
``Utilities'' menu, Section \ref{sec:versinf}.

\NextFile{update-step-4.html}

\section{Step 4: Enjoy!}

After you have finished updating Sailfish, there is generally nothing else
you need to do: you are ready to resume printing.  Again, the release notes
for the new version you have installed may contain special directions,
but generally you do not need to do anything.
